\documentclass[11pt]{article}
\usepackage{algorithm} %format of the algorithm 
\usepackage{algorithmic} %format of the algorithm 
\usepackage{multirow} %multirow for format of table 
\usepackage{amsmath} 
\usepackage{xcolor}
\usepackage{CJKutf8}
\usepackage{floatflt}
\renewcommand{\algorithmicrequire}{\textbf{Input:}} 
\renewcommand{\algorithmicensure}{\textbf{Output:}}
\title{实验二 决策树的实现}
\author{18308133 刘显彬}
\date{\today}
\begin{document}
\begin{CJK}{UTF8}{gbsn}
\maketitle
\section{伪代码}
\begin{algorithm}
\caption{PLApredict(X,W,b)}
\label{alg1}
\begin{algorithmic}
\REQUIRE $X:$输入数据, $W:$	权重, $b:$常数偏置
\ENSURE $Y$:预测值
\STATE reurn Sign(W*X+b)

\end{algorithmic}
	
\end{algorithm}

\begin{algorithm}

\caption{PLAtrain(X, l, W, b, iters, $\eta$)}\label{alg2}
\begin{algorithmic}
\REQUIRE $X$: 输入\ ;$l$: 标签;\ $iters$:迭代次数;\ $\eta$:学习率
\ENSURE $W,b$ : 更新后的W,b
\STATE $N \gets len(X)$
\STATE $iter,i \gets 0$
\WHILE{$iter \le iters$}
	\STATE $y \gets label[i]$
	\STATE $x \gets X[i]$
	\STATE //找到误分类点
	\IF{predict(x,W,b) != y}
	\STATE $iter \gets iter+1$
	\STATE //对W进行梯度下降更新
	\STATE $dW \gets -y*x$
	\STATE $W \gets W-\eta *dW$
	\ENDIF
	\STATE $i \gets (i+1)\%N$
\ENDWHILE
\STATE return W,b
\end{algorithmic}
\end{algorithm}


\begin{algorithm}
\caption{LRpredict(X, W, b)}\label{alg3}
\begin{algorithmic}
\REQUIRE $X:$输入数据, $W:$	权重, $b:$常数偏置
\ENSURE $Y$:预测值
\STATE $Y1\gets W*X+b$
\STATE $Y\gets \frac{1}{1+e^{-Y1}}$
\STATE return Y
\end{algorithmic}
\end{algorithm}



\begin{algorithm}

\caption{LRtrain(X, l, W, b, iters, $\eta$)}\label{alg4}
\begin{algorithmic}
\REQUIRE $X$: 输入\ ;$l$: 标签;\ $iters$:迭代次数;\ $\eta$:学习率
\ENSURE $W,b$ : 更新后的W,b
\STATE $N \gets len(X)$
\STATE $iter,i \gets 0$
\FOR{$iter \gets 0$ to iters}
	\STATE $p \gets$ predict(X,W,b)
	\STATE //对W进行梯度下降更新
	\STATE $dW\gets \sum_{i=0}^{N}X_i*(-l_i+p_i)$
	\STATE $dW \gets \frac{dW}{N	}$
	\STATE $W \gets W-\eta *dW$
\ENDFOR
\STATE return W,b
\end{algorithmic}
\end{algorithm}




\begin{algorithm}
\caption{Gini(d)}\label{alg8}
\begin{algorithmic}
\REQUIRE $d$: dataset
\ENSURE $gini$

\STATE // 分成不同的类
\STATE $spData \gets $ split $d$ with different label
\STATE $N \gets len(d)$
\STATE 计算这些类的比重
\STATE $freqs \gets \frac{len(dset)}{N}\ \forall dset \in spData$ \newline

\STATE return $1-\sum_{freq\in freqs}{freq^2}$
\end{algorithmic}
\end{algorithm}

\begin{algorithm}
\caption{Gini(d, attr)}\label{alg5}
\begin{algorithmic}
\REQUIRE $d$: dataset, $attr$: split attr
\ENSURE $gini$

\STATE // 像前一个算法一样分裂数据集
\STATE $spData \gets $ split $d$ with val $\in attr$
\STATE $N \gets len(d)$
\STATE $freqs \gets \frac{len(dset)}{N}\ \forall dset \in spData$ \newline
\STATE $gini \gets 0$
\FOR{$i \gets 0\ to\ len(freqs)$}{\label{get HD_A}
	\STATE $gini$ += $freqs[i]*Gini(spData[i])$;
		}
\ENDFOR
\STATE return $gini$
\end{algorithmic}
\end{algorithm}

%% decision 
\begin{algorithm}
\caption{buildTree(root, d, alg)}\label{alg6}
\begin{algorithmic}
\REQUIRE $d$: dataset, $root$:决策树根, $alg$:计算信息熵的算法
\IF{$d.attr == null$ or only one label in d.labels}
\STATE 这是一片叶子
\STATE root['attr']='leaf', root['val']=vote\_max(d.labels)
\STATE return root
\ELSE
\STATE $best \gets -inf$;
\STATE $bestattr \gets ''$;
\STATE //根据给定的算法找出最优的属性
\FORALL{$attr \in d.attr$}{
	\STATE $best \gets alg(d, attr)$;
	\STATE $bestattr \gets argmax(best,attr)$;

}
\ENDFOR \newline
\STATE //然后对最优属性进行分裂,对子节点进行迭代
\STATE $root['attr']\gets bestattr$
\STATE $spData \gets$ d splitted by bestattr;
\FOR{$sub \in spData$}{
	\STATE $root['val']\gets buildTree(root['val'], sub, alg)$;
}
\ENDFOR 
\ENDIF
\ENSURE $root$

\end{algorithmic}
\end{algorithm}

\begin{algorithm}
\begin{algorithmic}
\caption{predict(root, data)}
\REQUIRE root:  决策树树根, data: 待预测的数据
\ENSURE predVal: 预测值
\STATE $cur \gets root$
\STATE //当前不是叶子时,进行搜索
\STATE $attr \gets cur['attr']$
\WHILE{$attr$ != $'leaf'$}
\STATE //进入data[attr]对应的一支分枝
\STATE	$cur \gets cur['val'][data[attr]]$
\STATE  $attr \gets cur['attr']$
\ENDWHILE
\STATE //返回叶子的预测label
\STATE return cur['val']
\end{algorithmic}
\end{algorithm}



\end{CJK}
\end{document}
