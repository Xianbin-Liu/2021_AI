\documentclass[11pt]{article}
\usepackage{algorithm} %format of the algorithm 
\usepackage{algorithmic} %format of the algorithm 
\usepackage{multirow} %multirow for format of table 
\usepackage{amsmath} 
\usepackage{xcolor}
\usepackage{CJKutf8}
\usepackage{floatflt}
\renewcommand{\algorithmicrequire}{\textbf{Input:}} 
\renewcommand{\algorithmicensure}{\textbf{Output:}}
\title{实验二 决策树的实现}
\author{18308133 刘显彬}
\date{\today}
\begin{document}
\begin{CJK}{UTF8}{gbsn}
\maketitle
\section{实验原理}
\subsection{决策树}
\subsection{ID3模型}
\subsection{C4.5模型}
\subsection{Cart模型}
\subsection{决策树的形成过程}
\section{伪代码}
\textbf{算法1:计算给定数据集的熵值;}
\begin{algorithm}
\caption{cal-HD(l)}
\label{alg1}
\begin{algorithmic}
\REQUIRE $\l: label\ of \ dataset(D)$
\ENSURE $H(D)$
\STATE $labels \gets label \in l$
\STATE $freq \gets frequency\ of\ label$
\STATE $n \gets num\ of \ label$
\STATE HD = 0
\FOR{$i \gets 0\ to\ n$}{\label{for1}
	\STATE HD += $freq[i]*log(freq[i])$;}
\ENDFOR
\STATE reurn HD

\end{algorithmic}
	
\end{algorithm}

\textbf{算法2:计算信息增益}
\begin{algorithm}

\caption{Gain(d, attr, HD)}\label{alg2}
\begin{algorithmic}
\REQUIRE $d$: dataset, $attr$: split attr, $HD$: entropy of $d$
\ENSURE $gain$
\STATE // 分裂数据
\STATE $values \gets value \in d[attr]$;


\STATE $N \gets len(d)$;
\STATE set $spData = $ empty dict;
\FOR {$v$ such that $v \in values$}{\label{splitData}
	\STATE $spData[v] = subD, where subD[attr]=='v' and subD \in d$
	 
}
\ENDFOR
\STATE $H(D)_A$=0

\FOR{$sub$ such that $sub \in spData$}{\label{get HD_A}
	\STATE // 计算各个子数据集的熵,并求和
	\STATE $H(D)_A$ += $\frac{len(sub)}{N}*cal\_HD(sub)$;
		}
\ENDFOR
\STATE return $HD-H(D)_A$
\end{algorithmic}
\end{algorithm}


\begin{algorithm}
\caption{GainRatio(d, attr, HD)}\label{alg3}
\begin{algorithmic}
\REQUIRE $d$: dataset, $attr$: split attr, $HD$: entropy of $d$
\ENSURE $gainRatio$
\STATE // 像前一个算法一样分裂数据集
\STATE $spData \gets $ split $d$ with val $\in attr$
\STATE // 计算$SplitInfo$
\STATE $splitInfo \gets 0$
\STATE $N \gets len(d)$;
\FOR{$sub$ such that $sub \in spData$}
\STATE $SplitInfo$ += $\frac{len(sub)}{N}*log(\frac{len(sub)}{N})$
\ENDFOR
\STATE return $\frac{Gain(d,attr,HD)}{SplitInfo}$
\end{algorithmic}
\end{algorithm}


\begin{algorithm}
\caption{Gini(d)}\label{alg4}
\begin{algorithmic}
\REQUIRE $d$: dataset
\ENSURE $gini$

\STATE // 分成不同的类
\STATE $spData \gets $ split $d$ with different label
\STATE $N \gets len(d)$
\STATE 计算这些类的比重
\STATE $freqs \gets \frac{len(dset)}{N}\ \forall dset \in spData$ \newline

\STATE return $1-\sum_{freq\in freqs}{freq^2}$
\end{algorithmic}
\end{algorithm}

\begin{algorithm}
\caption{Gini(d, attr)}\label{alg5}
\begin{algorithmic}
\REQUIRE $d$: dataset, $attr$: split attr
\ENSURE $gini$

\STATE // 像前一个算法一样分裂数据集
\STATE $spData \gets $ split $d$ with val $\in attr$
\STATE $N \gets len(d)$
\STATE $freqs \gets \frac{len(dset)}{N}\ \forall dset \in spData$ \newline
\STATE $gini \gets 0$
\FOR{$i \gets 0\ to\ len(freqs)$}{\label{get HD_A}
	\STATE $gini$ += $freqs[i]*Gini(spData[i])$;
		}
\ENDFOR
\STATE return $gini$
\end{algorithmic}
\end{algorithm}

%% decision 
\begin{algorithm}
\caption{buildTree(root, d, alg)}\label{alg6}
\begin{algorithmic}
\REQUIRE $d$: dataset, $root$:决策树根, $alg$:计算信息熵的算法
\IF{$d.attr == null$ or only one label in d.labels}
\STATE 这是一片叶子
\STATE root['attr']='leaf', root['val']=vote\_max(d.labels)
\STATE return root
\ELSE
\STATE $best \gets -inf$;
\STATE $bestattr \gets ''$;
\STATE //根据给定的算法找出最优的属性
\FORALL{$attr \in d.attr$}{
	\STATE $best \gets alg(d, attr)$;
	\STATE $bestattr \gets argmax(best,attr)$;

}
\ENDFOR \newline
\STATE //然后对最优属性进行分裂,对子节点进行迭代
\STATE $root['attr']\gets bestattr$
\STATE $spData \gets$ d splitted by bestattr;
\FOR{$sub \in spData$}{
	\STATE $root['val']\gets buildTree(root['val'], sub, alg)$;
}
\ENDFOR 
\ENDIF
\ENSURE $root$

\end{algorithmic}
\end{algorithm}

\begin{algorithm}
\begin{algorithmic}
\caption{predict(root, data)}
\REQUIRE root:  决策树树根, data: 待预测的数据
\ENSURE predVal: 预测值
\STATE $cur \gets root$
\STATE //当前不是叶子时,进行搜索
\STATE $attr \gets cur['attr']$
\WHILE{$attr$ != $'leaf'$}
\STATE //进入data[attr]对应的一支分枝
\STATE	$cur \gets cur['val'][data[attr]]$
\STATE  $attr \gets cur['attr']$
\ENDWHILE
\STATE //返回叶子的预测label
\STATE return cur['val']
\end{algorithmic}
\end{algorithm}

\section{关键代码分析}
\section{实验结果和分析}
\section{思考题}
\section{遇到的问题和解决}
\section{Reference}


\end{CJK}
\end{document}
